\documentclass{article}

\usepackage{amsthm,amsmath,amssymb}
\usepackage[all]{xy}
\newcommand{\Q}{\mathbb Q}
\newcommand{\C}{\mathbb C}
\newcommand{\inv}{^{-1}}

\title{Solutions to Homework 4}
\author{Nathan Grigg}
\date{}

\begin{document}
\maketitle
{\noindent\bf Section 4, Exercises 1, 2, 3, and 4.}

\paragraph{Exercise 1.}  Let $G$ be a group of automorphisms of a field $K$. Prove that the fixed elements $K^G$ form a subfield of $K$.

\begin{proof}
By definition, $K^G\subseteq K$. Clearly $0,1\in K^G$. Suppose that $\alpha$ and $\beta$ are in $K^G$. Then for any $\varphi\in G$, we have $\varphi(\alpha)=\alpha$ and $\varphi(\beta)=\beta$. Then for any $\varphi\in G$, we have
\begin{align*}
\varphi(\alpha+\beta)&= \varphi(\alpha)+\varphi(\beta)=\alpha+\beta &
\varphi(\alpha-\beta)&= \varphi(\alpha)-\varphi(\beta)=\alpha-\beta \\
\varphi(\alpha\beta)&= \varphi(\alpha)\varphi(\beta)=\alpha\beta &
\varphi(\alpha/\beta)&= \varphi(\alpha)/\varphi(\beta)=\alpha/\beta \\
\end{align*}
All the other field axioms (associativity, commutativity,
distributivity) are met since $K^G$ is a subset of $K$. Hence $K^G$ is a field.
\end{proof}

\paragraph{Exercise 2.} Let $\alpha=\sqrt[3]{2}, \zeta=\frac12(-1+\sqrt{-3}), \beta=\alpha\zeta$.
\begin{itemize}
\item[{\bf(a)}] Prove that for all $c\in \Q$, $\gamma=\alpha+c\beta$ is a root of a sixth-degree polynomial of the form $x^6+ax^3+b$.
\item[{\bf(b)}] Prove that the irreducible polynomial for $\alpha+\beta$ is cubic.
\item[{\bf(c)}] Prove that $\alpha-\beta$ has degree 6 over $\Q$.
\end{itemize}

\begin{proof} \hfill 

\begin{itemize}
\item[{\bf(a)}] For this part, we will use the fact that $\zeta=1/2+i\sqrt3/2$ and $\zeta^2=\overline\zeta$. Also, $\zeta^3=1$. We compute
\begin{align*}
\gamma^3&=(\alpha+c\beta)^3=(\alpha(1+c\zeta))^3\\
&=\alpha^3(1+3c\zeta+3c^2\zeta^2+c^3)\\
&=2(1+c^3+3c(\zeta+c\overline\zeta))\\
&=2(1+c^3+(3c/2)(i\sqrt3(1-c)-(1+c)))\,.
\end{align*}
This is clearly in the field $\Q(i\sqrt3)$, which is a degree 2
extension of $\Q$. Thus $\gamma^3$ must satisfy a quadratic
polynomial, which means $\gamma$ satisfies a polynomial of
the form $x^6+ax^3+b$. 

\item[{\bf(b)}] Note that $\zeta^2+\zeta+1=0$, so $\alpha+\beta=\alpha(1+\zeta)=-\alpha\zeta^2$. This is a root of $x^3+2$, since
$(-\alpha\zeta^2)^3+2=-2+2=0$. This is an irreducible polynomial.

\item[{\bf(c)}] A straightforward calculation shows that $(1-\zeta)^6=-27$. So
$$(\alpha-\beta)^6=\alpha^6(1-\zeta)^6=4(-27)=-108\,.$$
Therefore $(\alpha-\beta)$ is a root of the polynomial $x^6+108$,
which is irreducible by Eisenstein's criterion on any prime other than
2 or 3. 
\end{itemize}
\end{proof}
\noindent Note: This last calculation shows that the splitting field $\Q(\alpha,\beta)$ of $x^3-2$ can also be expressed as $\Q(\alpha-\beta)$.

\paragraph{Exercise 3.} For each of the following sets of automorphisms of the field of rational functions $\mathbb C(y)$, determine the group of automorhpism which they generate, and determine the fixed field explicitly.\\
{\bf(a)} $\sigma(y) = y\inv$\ \
{\bf(b)} $\sigma(y)=iy$\ \
{\bf(c)} $\sigma(y)=-y, \tau(y)=y\inv$\\
{\bf(d)} $\sigma(y)=\zeta, \tau(y)=y\inv$, where $\zeta=e^{2\pi i/3}$ \ \
{\bf(e)} $\sigma(y)=iy, \tau(y)=y\inv$

\begin{proof}[Solution]\hfill

\begin{itemize}
\item[{\bf(a)}] Clearly $\sigma^2$ is the identity, so the group of
  automorphisms is the cyclic group of order 2, generated by $\sigma$.  
Note that $w=y+\frac1y$ is fixed by $\sigma$, so we have
$\C(w)\subseteq \C(y)^G$. Since $[\C(y):\C(y)^G]=2$, this means that
$[\C(y):\C(w)]\geq 2$, with equality if and only if $\C(w)=\C(y)^G$. 
Then note that $y$ is a root of the polynomial $x^2-wx+1$ in $\C(w)[x]$, so
$[\C(y):\C(w)]\leq 2$, which means that $\C(w)=\C(y)^G$, as desired.
So the fixed field is $\C(y+\frac1y)$.

\item[{\bf(b)}] Here $\sigma$ has order 4, so the group of
  automorphisms is the cyclic group of order 4. Now, $w=y^4$ is fixed
  by $\sigma$ (hence by all of $G$). And $y$ is a root of the
  polynomial $x^4-w$, so the fixed field is equal to $\C(y^4)$. 


\item[{\bf(c)}] Both $\sigma$ and $\tau$ are order 2 and commute, so
  the group of autmormophisms is isomorphic to $V_4$. The element
  $w=y^2+1/y^2$ is fixed by both $\sigma$ and $\tau$, and $y$ is a
  root of 
  the polynomial $x^4-wx^2+1$, so the fixed field is $\C(y^2+1/y^2)$.


\item[{\bf(d)}] Here we have an element of order 2 and an element of
  order 3. Also, $\sigma$ and $\tau$ satisfy $\sigma \tau = \tau
  \sigma^2$, so the group is isomorphic to
  $S_3$. Note that $\sigma$ and $\tau$ both fix $w=y^3+1/y^3$, and $y$
  is a root of $x^6-wx^3+1$, so the fixed field is $\C(y^3+1/y^3)$. 


\item[{\bf(e)}] Here we have an element of order 2 and an element of
  order 4 with the relation $\sigma\tau=\tau\sigma^3$, so we have
  $D_4$. Since $\sigma$ and $\tau$ both fix $w=y^4+1/y^4$, and $y$ is
  a  root of $x^8-wx^4+1$, the fixed field is $\C(y^4+1/y^4)$.

\end{itemize}
\end{proof}



\paragraph{Exercise 4.} Show that the group $G$ generated by the automorphisms
$\sigma(y)=(y+i)/(y-i), \tau(y)=i(y-1)/(y+1)$ of $\mathbb C(y)$
is isomorphic to the alternating group $A_4$. Determine
the fixed field of this group.  



\begin{proof}[Solution]
For simplicity, write $A=\sigma(y)$ and $B=\tau(y)$. Some calculation
shows
\begin{align*}
\sigma(A)&=-1/B & \sigma(B)&=-1/y &
\tau(A)&=-y & \tau(B)&=-A
\end{align*}
Also note that $\sigma^3=\tau^3=1$.
From these calulcations, we see $\sigma$ and $\tau$ act as follows:
\newcommand{\ds}{\displaystyle}
\begin{equation*}
\xymatrix{
  \ds\frac{y}{AB} \ar@{|->}[r]^{\sigma}
    & ABy      \ar@{|->}[r]^{\sigma}
    & \ds\frac{A}{By} \ar@{|->}[r]
    & \ds\frac{y}{AB},
    & \sigma\text{ fixes }\ds\frac{B}{Ay}\\
  \ds\frac{y}{AB} \ar@{|->}[r]^{\tau}
    & \ds\frac{B}{Ay}    \ar@{|->}[r]^{\tau}
    & \ds\frac{A}{By} \ar@{|->}[r]
   & \ds\frac{y}{AB},
    & \tau\text{ fixes }ABy.
}
\end{equation*}
Since $G$ permutes elements of the set
$\ds\left\{\frac{y}{AB},ABy,\frac{A}{By},\frac{B}{Ay}\right\}$ and is
generated by even permutations, $G$ is a subgroup of $A_4$. Note that
$G$ has two distinct subgroups of order 3. From this we get that $|G|$
has to be either 6 or 12. But $|G|$ cannot be 6, because in that case
a subgroup of order 3 is normal, since it has index 2.
All 3-Sylow subgroups are conjugate, and all conjugates of a 
normal subgroup are equal, so there would only only be one such subgroup.
Therefore $G=A_4$. (A simpler, but less cool sounding way to prove
this last bit would be to show that $G$ has at least 7 different
elements, which means it must 
have 12 and we are done.)

Another way to prove this would be to write down all the elements, but
then you would also have to prove that there are no others, which
would be tedious. If you were wondering, the elements of $G$ are:
$$\{1,\sigma,\tau,\sigma^2,\sigma\tau,\tau\sigma,\tau^2,\sigma^2\tau
     ,\sigma\tau^2,\tau\sigma^2,\tau^2\sigma,\tau\sigma^2\tau\}$$

	 The fixed field of $G$ is given $\C(w)$, where
\begin{align*}
w&=\left(y^2+\frac{1}{y^2}\right)\left(A^2+\frac{1}{A^2}\right)
\left(B^2-\frac{1}{B^2}\right)\\
&= {\frac {-4\,{y}^{12}+132\,{y}^{8}+132\,{y}^{4}-4}{{y}^{10}-2\,{y}^{6}+
{y}^{2}}}
\end{align*}
It is clear that $\sigma$ and $\tau$ fix $w$, and $y$ is a root of the
degree 12 polynomial
$$f(x)=4\,{x}^{12}-132\,{x}^{8}-132\,{x}^{4}+ \left( {x}^{10}-2\,{x}^{6}+{x}^
{2} \right) w+4\in\C(w)[x],$$
so this is the fixed field we are looking for.
\end{proof}
\end{document}
