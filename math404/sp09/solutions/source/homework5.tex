\documentclass{article}

\usepackage{amsthm,amsmath,amssymb}
\usepackage[all]{xy}
\newcommand{\Q}{\mathbb Q}
\newcommand{\C}{\mathbb C}
\newcommand{\R}{\mathbb R}
\newcommand{\inv}{^{-1}}

\title{Solutions to Homework 5}
\author{Nathan Grigg}
\date{}

\begin{document}
\maketitle
{\noindent\bf Section 5, Exercises 1, 3, 4, 7, 10, and 11}

\paragraph{Exercise 1.}  Let $K=\Q(\alpha)$, where $\alpha$ is a root of the polynomial $x^3+2x+1$, and let $g(x)=x^3+x+1$. Does $g(x)$ have a root in $K$?

\begin{proof}[Solution]
The answer is no, which we will prove by contradiction. 
Let $L$ denote the splitting field of $f$.
Assume that $\beta\in K$ is a root of $g(x)$. 
Then $\beta\in L$. But $g$ is irreducible, which means that
$\beta\not\in \Q$. Since $L/\Q$ is Galois, there must be some element
of $G(L/\Q)$ that does not fix $\beta$. But this automorphism must
take $\beta$ to some other root of $g$, which means that this other
root of $g$ must be an element of $L$. So in fact all the roots of $g$
are in $L$.

In particular, the square roots of the discrimants of $f$ and $g$
(i.e. $\sqrt{-59}$ and $\sqrt{-31}$) are in $L$. But this means that
$L$ has two intermediate fields of degree $2$ over $\Q$, which is
impossible since $G(L/\Q)=S_3$, which
 has only one subgroup of index 2. So we have a
contradiction, which means that our assumsion was not true. So $g$ has
no roots in $K$.
\end{proof}

\paragraph{Exercise 3.} Let $G$ be a finite group. Prove that there
exists a field $F$ and a Galois extension $K$ of $F$ whose Galois
group is $G$. 

\begin{proof}
First note that by Caylee's theorem, $G\cong H$, where $H$ is a subgroup
of $S_n$ for some $n$. Let $K=\Q(x_1,\ldots,x_n)$ be the field of rational
functions in $n$ variables. For each $\sigma\in S_n$ there is an
automorphism $\tilde \sigma$ of $K$ that sends $x_i$ to $x_{\sigma
  (i)}$ for each $i$. It is clear that $\tilde\sigma \tilde\tau
(\alpha ) = \widetilde{\sigma\tau}(\alpha)$ for all $\alpha\in
  K$. Also, $\tilde\sigma=\tilde\tau$ implies $\sigma=\tau$, 
so $S_n$ is a subgroup of the group of automorphisms of $K$. Let $L$
be the fixed field of $S_n$, then $K/L$ is Galois, with
$G(K/L)=S_n$. Finally, let $F=K^H$. By the main Galois theorem,
$G(K/F)=H$, which is isomorphic to $G$, as required.
\end{proof}

\paragraph{Exercise 4.} Assume it is known that $\pi$ and $e$ are
transcendental numbers. Let $K$ be the plitting field of the
polynomial $f(x)=x^3+\pi x+6$ over the field $F=\Q(\pi)$. Prove that
$[K:F]=6$ and that $K$ is isomorphic to the splitting field of
$f(x)=x^3+ex+6$ over $\Q(e)$.

\begin{proof}
First, we show that $f(x)$ is irreducible over $\Q(\pi)$. If $f$ is
reducible then since it is cubic it must have a root $\alpha$. This
gives us an equation $\alpha^3+\pi \alpha+6=0$, where $\alpha\in
\Q(\pi)$. We can take this equation, change all the $\pi$'s to $x$'s,
and clear denominators, and we get a polynomial in $\Q[x]$ that has
$\pi$ as a root, which is a contradiction since $\pi$ is
transcendental. Therefore $f$ is irreducible. Finally, the square root
of the discrimant is $\sqrt{-972-4\pi^3}$, which is not in $\R$, much
less in $\Q(\pi)$. So the Galois group of $f$ is $S_3$, which means
$[K:F]=6$. 

Next, there is an isomorphism of the fields $\Q(\pi)$ and $\Q(e)$
which takes $\pi$ to $e$, which means it takes $f$ to $g$. Then by
Proposition~(5.2), the splitting field of $f$ is isomorphic to the
splitting field of $g$, as desired.
\end{proof}

\paragraph{Exercise 7.} Let $f$ be an irreducible cubic polynomial
over $\mathbb Q$ whose Galois group is $S_3$. Determine the possible
Galois groups of the polynomial $(x^3-1)\cdot f(x)$. 

\begin{proof}[Solution]
Let $K$ be the splitting field of $f$, and let $\omega$ be a non-real
root of $x^3-1$. We are asked to determine the possibilities for
$G=G(K(\omega)/\Q)$. First, note that if $\omega\in K$, then
we have  $G=S_3$. 

If $\omega \not\in K$, then we have the following field diagram:
\begin{equation*}
\xymatrix{
&K(\omega)\ar@{-}[dl]_{6} \ar@{-}[dr]^{2}& \\
\Q(\omega) && K \\
&\Q\ar@{-}[ul]_2 \ar@{-}[ur]^{6}
}
\end{equation*}
Now, $\Q(\omega)/\Q$ is a Galois extension, 
since $\Q(\omega)$ is the splitting field of a polynomial.
By the Galois correspondence, there is a normal subgroup $H$ of $G$
such that $G/H\cong G(\Q(\omega)/\Q)\cong C_2$.
For the same reasons, $K/\Q$ is Galois, and there is a normal subgroup
$H'$ such that $G/H'\cong G(K/\Q)\cong S_3$. A basic group theory
result tells us that since $H$ and $H'$ are normal, the product $HH'$
is a subgroup of $G$, so it must correspond to a subfield of
$K(\omega)$. To find out which, we compute the fixed field of
$HH'$. But $H$ and $H'$ are both subgroups of $HH'$, so if $\alpha$ is
fixed by $HH'$ it must be fixed by $H$ and $H'$ as well, which means
it must be in $\Q(\omega)$ (the fixed field of $H$) and $K$ (the fixed
field of $H'$), which means it must be in $\Q$. But $\Q$ is the fixed
field of $G$, which means that $HH'=G$. Now we know that $|H\cap H'|
= |H||H'|/|HH'| = 1$. This fact, together with the fact that $H$ and
$H'$ are normal subgroups, means that $G=HH'\cong H\times H'$.
But then $H\cong G/H'\cong S_3$ and $H'\cong G/H\cong C_2$, so we have
$G\cong S_3\times C_2$. This is isomorphic to the dihedral group of
order 12.
\end{proof}

\paragraph{Exercise 10.} Let $K$ be a splitting field of an
irreducible cubic polynomial $f(x)$ over a field $F$ whose Galois
group is $S_3$. Determine the group $G=G(F(\alpha)/F)$. 

\begin{proof}[Solution]
Since $F(\alpha)\subseteq K$, with $[K:F(\alpha)]=2$, 
$F(\alpha)$ corresponds to a subgroup of $S_3$ of order $2$. We know
that the subgroups of $S_3$ of order $2$ are not normal subgroups, so
therefore $F(\alpha)/F$ is not Galois, which implies that
$|G|\neq 3$. But we know that the order of $G$ divides $3$, which
means that $|G|=1$, and $G$ is the trivial group.
\end{proof}


\paragraph{Exercise 11.} Let $K/F$ be a Galois extension whose Galois
group is the symmnetric group $S_3$. Is it true that $K$ is the
splitting field of an irreducible cubic polynomial over $F$? 

\begin{proof}[Solution]
Yes. Let $H$ be a subgroup of $S_3$ of order 2. Let $K^H$ be the fixed
field of $H$, and let $\alpha$ be such that $K^H=F(\alpha)$. Then
$[F(\alpha):F]=3$, so $\alpha$ is the root of a degree 3 polynomial in
$F[x]$, which I will call $f(x)$. I claim that $K$ is the splitting
field of $f$. 

Since $f$ has one root of $K$, it has all its roots in $K$ by
the argument we gave in Exercise~1. Also, $H$ is not normal in $S_3$,
so $F(\alpha)/F$ is not Galois, therefore $F(\alpha)$ is not the
splitting field of $f$. So the splitting field of $f$ contains but is
not equal to $F(\alpha)$ and is contained in $K$. By the Galois
theorem, the splitting field of $f$ corresponds to a subgroup of $S_3$
that is contained in but not equal to $H$. But $|H|=2$, so the
splitting field must correspond to the trivial subgroup, which means
the fixed field is $K$, as desired.
\end{proof}

\end{document}
