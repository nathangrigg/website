\documentclass{article}

\usepackage{amsthm,amsmath,amssymb}
\newcommand{\Q}{\mathbb Q}
\newcommand{\C}{\mathbb C}
\newcommand{\R}{\mathbb R}
\newcommand{\inv}{^{-1}}

\title{Solutions to Homework 6}
\author{Nathan Grigg}
\date{}

\begin{document}
\maketitle
{\noindent\bf Section 7, Problems 1, 3, and 5}

\paragraph{Exercise 1.} Suppose that for some integer $n$, $F$ contains the $n$th roots of unity, and $K/F$ is a Galois extension of the form $K=F(\alpha)$, where $\alpha^n\in F$. What can you say about the Galois group $G=G(K/F)$?

\begin{proof}[Solution]
We can say that $G(K/F)$ is cyclic. Note that $\alpha$ is a root of $f(x)=x^n-\alpha^n\in F[x]$. Let $\omega$ be a primitive $n$th root of unity; then $\alpha\omega^i$ is a root of $f(x)$ for each $i$ between 0 and $n-1$. These numbers are all distinct, so these are all the roots of $f(x)$, which means that we have
$$f(x)=(x-\alpha)(x-\alpha\omega)\cdots(x-\alpha\omega^{n-1}).$$
Now, any $F$-automorphism of $K$ is determined by where it sends $\alpha$, and it must send roots of $f(x)$ to other roots of $f(x)$, so it is of the form $\sigma_i(\alpha)=\alpha\omega^i$ for some $i$. Note that $\sigma_i\circ \sigma_j=\sigma_{i+j}$, so there is a (clearly injective) homomorphism from $G$ to $C_n$ given by $\sigma_i\mapsto i$. Thus $G$ is isomorphic to a subgroup of $C_n$, which means that $G$ itself is also cyclic. 
\end{proof}


\paragraph{Exercise 3.} Let $F$ be a subfield of $\C$ which contains $i$, and let $K$ be a Galois extension of $F$ whose Galois group is $C_4$. Is it true that $K$ has the form $F(\alpha)$, where $\alpha^4\in F$?

\begin{proof}[Solution]
Yes. Let $\sigma$ be a generator of $G(K/F)$. Then if $\beta$ is an eigenvector of $\sigma$ with eigenvalue $\lambda$, we have $\beta=\sigma^4(\beta)=\lambda^4\beta$. So $\lambda^4=1$.

Then since $\sigma$ has finite order, it is diagonalizable, i.e., there is a basis for which the matrix for $\sigma$ is diagonal whose entries are eigenvalues of $\sigma$. Suppose that $\pm i$ are not eigenvalues for $\sigma$, then the matrix for $\sigma$ just has $\pm1$ down the diagonal, which means that $\sigma^2$ is the identity. This is a contradiction, so $\lambda$ is an eigenvalue for $\sigma$ for either $\lambda=i$ or $\lambda=-i$. Let $\gamma$ be the corresponding eigenvector. Then
$$\gamma\sigma(\gamma)\sigma^2(\gamma)\sigma^3(\gamma)=\lambda\lambda^2\lambda^3\gamma^4=-\gamma^4.$$
Since this is fixed by $\sigma$, it is in $F$, so $\gamma^4\in F$.
Also, $\sigma^k(\gamma)\neq \gamma$ for $k=1,2,3$. 
Hence $\gamma$ is not fixed by any subgroup of $\langle \sigma
\rangle$, which implies that 
$K=F(\gamma)$.
\end{proof}

\paragraph{Exercise 5.} Let $K$ be  a splitting field of an
irreducible polynomial $f(x)\in F[x]$ of degree $p$ whose Galois group
is a cyclic group of order $p$ generated by $\sigma$, and suppose that
$F$ contains the $p$th root of unity $\zeta=\zeta_p$. Show that there
is an ordering $\alpha_1,\alpha_2,\ldots,\alpha_p$ of the roots of $f$
such that
$$\beta=\alpha_1+\zeta^\nu\alpha_2+\zeta^{2\nu}\alpha_3+\cdots+\zeta^{(p-1)\nu}\alpha_p$$
is an eigenvector of $\sigma$, with eigenvalue $\zeta^{-v}$, unless it
is zero. 
 
\begin{proof}
Let $\alpha$ be a root of  $f$ and let $\alpha_i=\sigma^{i-1}(\alpha)$ for each $i$ between 1 and $p$. Then we can write
$$\beta=\sum_{i=0}^{p-1} \zeta^{\nu i}\sigma^i(\alpha),$$
and we have
$$\sigma(\beta)=\sigma\left(\sum_{i=0}^{p-1} \zeta^{\nu i}\sigma^i(\alpha)\right)=\sum_{i=0}^{p-1}\zeta^{\nu i}\sigma^{i+1}(\alpha)
=\sum_{j=1}^{p}\zeta^{-\nu}\zeta^{\nu j}\sigma^{j}(\alpha)=\zeta^{-\nu}\beta,$$
as desired.
\end{proof}


\end{document} 