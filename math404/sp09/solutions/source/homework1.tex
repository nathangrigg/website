\documentclass{article}

\usepackage{amsthm,amsmath,amssymb}

\title{Solutions to Homework 1}
\date{}

\begin{document}
\maketitle
{\noindent\bf Section 1, Exercises 1, 2, 3, and 5. }

\paragraph{Exercise 1.} Determine the irreducible polynomial for $i+\sqrt 2$ over $\mathbb Q$.
\begin{proof}[Solution]
The irreducible polynomial is $f(x)=x^4-2x^2+9$. To show that $f$ is indeed the irreducible polynomial of $i+\sqrt 2$, we must show that $i+\sqrt 2$ is a root of $f$ and that $f$ is irreducible over $\mathbb Q$. An easy calculation shows that $f(i+\sqrt 2)=0$, so the first part is done.

Suppose that $f$ is reducible. There are two possibilities: either $f$ is the product of a linear polynomial and a cubic polynomial, or $f$ is the product of two quadratic polynomials.
In the first case, $f(x)=0$ for some $x\in \mathbb Q$. This cannot be, since $f(x)=(x^2-1)^2+8$, which is positive for all real $x$. In the second case, $i+\sqrt 2$ must be a root of one of the factors of $f$. But $i+\sqrt 2$ cannot be the root of a quadratic polynomial, since by the quadratic formula, all roots of rational quadratic polynomials are of the form $a+\sqrt b$ for some $a,b\in \mathbb Q$. Hence $f$ must be irreducible.
\end{proof}

\paragraph{Exercise 2.} Prove that the set $(1,i,\sqrt 2, i\sqrt 2)$ is a basis for $\mathbb Q(i,\sqrt 2)$ over $\mathbb Q$.

\begin{proof}
Consider the set $S=\text{Span}(1,i,\sqrt 2,i\sqrt 2)\subseteq \mathbb Q(\sqrt 2,i)$. 
We see that $S$ is an integral domain and a finite dimensional vector space over $\mathbb Q$, so it is a field containing $\sqrt 2$ and $i$.  Since $\mathbb Q(\sqrt 2,i)$ is the smallest field containing $\sqrt 2$ and $i$, we have $\mathbb Q(\sqrt 2,i)\subseteq S$, so the two are equal. This shows that the potential basis is a spanning set.

On the other hand, since $i\not\in \mathbb Q(\sqrt 2)$, the degree of $\mathbb Q(\sqrt 2,i)/\mathbb Q(\sqrt 2)$ is 2. The degree of $\mathbb Q(\sqrt 2)/\mathbb Q$ is also 2, so the degree of $\mathbb Q(\sqrt 2,i)/\mathbb Q$ is 4. But 4 vectors can only span a 4-dimensional vector space if they are linearly independent, so the potential basis is indeed a basis.
\end{proof}

\paragraph{Exercise 3.} Determine the intermediate fields between $\mathbb Q$ and $\mathbb Q(\sqrt 2,\sqrt 3)$.
\begin{proof}[Solution]
There are five. The Galois group $G$ of the extension is $\{\text{id},\sigma,\tau,\sigma\tau\}$, where
\begin{align*}
\text{id}:&\sqrt 2\mapsto \sqrt 2 &
\sigma:& \sqrt 2\mapsto -\sqrt 2 &
\tau:& \sqrt 2\mapsto \sqrt 2 &
\sigma\tau:& \sqrt 2\mapsto -\sqrt 2 \\
&\sqrt 3\mapsto \sqrt 3 &
& \sqrt 3\mapsto \sqrt 3 &
& \sqrt 3\mapsto -\sqrt 3 &
& \sqrt 3\mapsto -\sqrt 3 .
\end{align*}
Since the degree of the field extension is 4 and we have 4 elements in the Galois group, the extension is Galois. So we can use the main Galois theorem, which says that there is a correspondence between the subgroups of $G$ and the intermediate fields of $\mathbb Q(\sqrt 2,\sqrt 3).$ Now $G$ is isomorphic to the Klein group, and has 5 subgroups. Each subgroup corresponds to an intermediate field which is the set of all elements fixed by everything in the subgroup:
\begin{align*}
\{\text{id}\}  &\leadsto  \mathbb Q(\sqrt 2,\sqrt 3)\\
\{\text{id},\sigma\}  &\leadsto  \mathbb Q(\sqrt 3)\\
\{\text{id},\tau\}  &\leadsto  \mathbb Q(\sqrt 2)\\
\{\text{id},\sigma\tau\}  &\leadsto  \mathbb Q(\sqrt 6)\\
\{\text{id},\sigma,\tau,\sigma\tau\}  &\leadsto  \mathbb Q
\end{align*}

\end{proof}

\paragraph{Exercise 5.} Prove that the automorphism of $\mathbb Q(\sqrt 2)$ sending $\sqrt 2$ to $-\sqrt 2$ is discontinuous.
\begin{proof}
If we call this automorphism $f$, note that $f(x)=x$ for all $x\in \mathbb Q$. Since $\mathbb Q$ is dense in $\mathbb Q(\sqrt 2)$, $f$ must be the identity on all of $\mathbb Q(\sqrt 2)$ if it is to be continuous. But $f(\sqrt 2)=-\sqrt 2$, so $f$ is not continuous. 

Another way to think about this is to consider a sequence $x_i$ of rational numbers that approach $\sqrt 2$. Then $f(x_i)$ approaches $\sqrt 2$ which is not equal to $f(\sqrt 2)$. Since $f$ does not preserve limits of sequences, it must not be continuous. 
\end{proof}
\end{document} 