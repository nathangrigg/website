\documentclass{article}

\usepackage{amsthm,amsmath,amssymb}
\newcommand{\Q}{\mathbb Q}

\title{Solutions to Homework 3}
\date{}

\begin{document}
\maketitle
{\noindent\bf Section 2, Exercises 2, 3, 4, and 6.}

\paragraph{Exercise 2.} Determine the Galois groups of the following polynomials.\\
\hbox{{\bf (a)} $x^3-2$} \ \ 
\hbox{{\bf (b)} $x^3+27x-4$} \ \ 
\hbox{{\bf (c)} $x^3+x+1$} \ \ 
\hbox{{\bf (d)} $x^3+3x+14$} \\ 
\hbox{{\bf (e)} $x^3-3x^2+1$} \ \ 
\hbox{{\bf (f)} $x^3-21x+7$} \ \ 
\hbox{{\bf (g)} $x^3+x^2-2x-1$} \\ 
\hbox{{\bf (h)} $x^3+x^2-2x+1$}

\begin{proof}[Solution]
We know that the Galois group of a polynomial permutes the roots of that polynomial, so the Galois group of a cubic must be a subgroup of $S_3$. Up to isomorphism, there are only 4 subgroups of $S_3$: the trivial group, $C_2$, $C_3$, and $S_3$. Since these all have different orders, we only need to find the degree of the Galois group and we will know that Galois group. 

There are four cases:
\begin{enumerate}
\item If $f$ splits into linear factors over $\Q$, then the splitting field has degree 1, so the Galois group is trivial.
\item If $f$ is reducible over $\Q$, but factors into a linear term and a quadratic term, then the splitting field has degree 2, so the Galois group is $C_2$.
\item If $f$ is irreducible over $\Q$ and the discriminant is a square in $\Q$, then the splitting field has degree 3, so the Galois group is $C_3$. 
\item If $f$ is irreducible over $\Q$ and the discriminant is not a square in $\Q$, then the splitting field has degree 6, so the Galois group is $S_3$.
\end{enumerate} 

To check irreducibility, we note that a cubic is reducible over some field if and only if it has a root in that field. Then we can use the rational roots theorem, which says that any rational root of a monic polynomial with constant term $r/s$ must be of the form $\pm a/b$, where $a$ divides $r$ and $b$ divides $s$. We compute the discriminant of $x^3+px+q$ by $-4p^3-27q^2$. For a general polynomial $x^3+ax^2+bx+c$, we use the substitution $x=y-a/3$ to get a polynomial in the special form above. We note that since the discriminant depends only on the differences between the roots, this substitution does not affect the discriminant.

{\bf(d)}: factors as $(x-2)(x^2-2x+7)$, and this second factor is irreducible, so the Galois group is $C_2$. 

{\bf (e), (f), (g)}: irreducible with discriminants 81, 35721, and 49, respectively, which are all squares in $\Q$. So the Galois group is $C_3$.

{\bf (a), (b), (c), (h)}: irreducible with discriminants $-108, -79164, -31, -31$, respectively, none of which is a square in $\Q$. So the Galois group is $S_3$.  
\end{proof}

\paragraph{Exercise 3.} Let $f$ be an irreducible cubic polynomial over $F$, and let $\delta$ be the square root of the discriminant of $f$. Porve that $f$ remains irreducible over the field $F(\delta)$. 

\begin{proof}
Suppose that $f$ is reducible over $F(\delta)$. Then since $f$ is a cubic, it must have a root $\alpha$ in the field $F(\delta)$. Then we can form a tower of fields
$$F\subseteq F(\alpha) \subseteq F(\delta).$$
But since $f$ is irreducible, $[F(\alpha):F]=3$, but $[F(\delta):F]$ is clearly 1 or 2. This is a contradiction. 
\end{proof}

\paragraph{Exercise 4.} Let $\alpha$ be a complex root of $f(x)=x^3+x+1$ over $\Q$, and let $K$ be a splitting field of this polynomial over $\Q$. 
\begin{itemize}
\item[\bf(a)] Is $\sqrt{-3}$ in $\Q(\alpha)$? Is it in $K$?
\item[\bf(b)] Prove that $\Q(\alpha)$ has no automorphism except the identity.
\end{itemize}

\begin{proof}[Solution]
The answer to $(a)$ is no to both questions. Suppose that $\sqrt{-3}$ were in $K$. Then since the square root of the discriminant (which we calcluated above as $\sqrt{-31}$) can be written in terms of the roots of $f$, it is also in $K$. This means that we have two intermediate fields $\Q(\sqrt{-3})$ and $\Q(\sqrt{-31})$ that are both degree 2 extensions of $\Q$. By the main Galois theorem, the intermediate fields of degree 2 are in correspondence with the index 2 (order 3) subgroups of the Galois group of $f$, which we calculated to be $S_3$. But $S_3$ only has two elements of order 3, and hence one subgroup of order 3, so this is a contradiction. Of course, this implies also that $\Q(\sqrt{-3})$ is not in $\Q(\alpha)$. 

For $(b)$ I will show that $\alpha$ is the only root of $f$ in $\Q(\alpha)$, which implies that there are no non-identity automorphisms of $\Q(\alpha)$. Suppose that another root $\beta$ of $f$ is in $\Q(\alpha)$. Then since the product of the three roots of $f$ is 1, the third root is equal to $1/(\alpha\beta)$, which must also be in $\Q(\alpha)$. Then $\Q(\alpha)$ must be the splitting field of $f$. Since $f$ is irreducible, $\Q(\alpha)$ must be a degree 3 extension, which contradicts the fact that the Galois group of $f$ is $S_3$. 
\end{proof}

\paragraph{Exercise 6.} Let $f\in \Q[x]$ be an irreducible cubic polynomial which has exactly one real root, and let $K$ be its splitting field over $\mathbb Q$. Prove that $[K:\Q]=6$. 

\begin{proof}
Let $\alpha$ be the real root of $f$. Since $f$ is irreducible, $\Q(\alpha)$ is a degree 3 extension of $\Q$. Since $\Q(\alpha)\subseteq \mathbb R$, it does not contain all the roots of $f$ and thus is not equal to $K$. Hence $[K:\Q]>[\Q(\alpha):\Q]=3$, but we also know that $[K:\Q]$ divides 6, so $[K:\Q]=6$, as desired. 
\end{proof}

\end{document}

