\documentclass{article}

\usepackage{amsthm,amsmath,amssymb}
\usepackage[all]{xy}
\newcommand{\Q}{\mathbb Q}

\title{Solutions to Homework 2}
\date{}

\newcommand{\Gal}{G}

\begin{document}
\maketitle
{\noindent\bf Section 1, Exercises 8, 10, 12, 13, 16, 17} (updated),
{\bf and 18.}

\paragraph{Exercise 8.} Let $\zeta=e^{2\pi i/5}.$
\begin{itemize}
\item[(a)] Prove that $K=\Q(\zeta)$ is a splitting field for the polynomial $f(x)=x^5-1$ over $\Q$, and determine the degree $[K:\Q]$.
\item[(b)] Prove that $K$ is a Galois extension of $\Q$, and determine its Galois group.
\end{itemize}

\begin{proof}[Solution]
For (a), note that $x^5-1=(x-1)(x-\zeta)(x-\zeta^2)(x-\zeta^3)(x-\zeta^4)$, so $f$ factors into linear factors in $K$. Also, $K$ is generated by the roots of $f(x)$, so $K$ is the splitting field of $f$.
Also, $g(x)=x^4+x^3+x^2+x+1$ is irreducible over $\Q$, since it is irreducible mod 2. It has $\zeta$ as a root, so it is the irreducible polynomial of $\zeta$. Thus $[K:\Q]=4$.

For (b), any automorphism of $K$ is completely determined by how it acts on $\zeta$. Since $g(x)$ defined above is irreducible, we can send $\zeta$ to any root of $g(x)$, i.e., to $\zeta$, $\zeta^2$, $\zeta^3$, or $\zeta^4$. These are the only possible automorphisms of $K$, and they are all different, so $|\Gal(K/\Q)|=4$. Hence $K/\Q$ is Galois. The Galois group is cyclic, generated by the automorphism $\zeta \mapsto \zeta^2$, since this element of the Galois group has order four.
\end{proof}

\paragraph{Exercise 10.} Let $K=\Q(\sqrt 2,\sqrt 3,\sqrt 5)$. Determine $[K:\Q]$, prove that $K$ is a Galois extension of $\Q$, and determine its Galois group.

\begin{proof}[Solution]
In the tower of fields $\Q\subseteq \Q(\sqrt 2)\subseteq \Q(\sqrt
2,\sqrt 3)\subseteq K$, each field is a proper subset of the one that
follows, and each extension is of degree at most 2, so each extension
must be of degree exactly 2. Hence $[K:\Q]=2\cdot 2\cdot 2=8.$

Elements of the Galois group are determined by where they send $\sqrt
2$, $\sqrt 3$, and $\sqrt 5$, and they must send each of these to
their negative. Since $\sqrt2$, $\sqrt3$, and $\sqrt 5$ are not roots
of the same irreducible polynomial, we can choose where each goes
independently of where another is mapped.  So
$$\Gal(K/\Q) = \{\text{id}, \sigma, \tau, \rho, \sigma\tau, \sigma\rho, \tau\rho, \sigma\tau\rho\}.$$
where $\sigma$, $\tau$, and $\rho$, map $\sqrt 2,\sqrt 3$, and $\sqrt 5$ to
$-\sqrt 2, -\sqrt 3$, and $-\sqrt 5$, respectively, and fix the other field
generators. In particular, $|\Gal(K/\Q)|=[K:\Q]=8$, so the extension
is Galois. Note $\Gal(K/\Q)\cong C_2\times C_2\times C_2$.
\end{proof}

\paragraph{Exercise 12.} Determine all automorphisms of the field
$\Q(\sqrt[3] 2)$.

\begin{proof}[Solution]
Note that $f(x)=x^3-2$ is a polynomial having $\sqrt[3]2$ as root, so any
automorphism of $\Q(\sqrt[3] 2)$ would have to take $\sqrt[3]2$ to
another root of $f$. But we know (since $f'(x)\geq 0$ for all $x$)
that $f$ has only one real root, so it must have two non-real
roots. But $\Q(\sqrt[3]2)\subseteq \mathbb R$, so these two roots are
not elements of $\Q(\sqrt[3]2)$. Hence any automorphism has to map
$\sqrt[3]2$ to itself, meaning that it is the identity.

This is an example of a field extension which is not Galois.
\end{proof}

\paragraph{Exercise 13.} Let $K/F$ be a finite extension. Prove that
the Galois group $G(K/F)$ is a finite group.
\begin{proof}
Since $K/F$ is a finite extension, $K$ is finitely generated as a field over $F$ (for example, it is generated by the elements of a basis).
So we can write $K=F(a_1,a_2,\ldots,a_n)$. Finite extensions are algebraic, so each $a_i$ is the root of a polynomial, say
of degree $d_i$. Then every transformation is determined by how it
acts on the set $\{a_1,\ldots,a_n\}$, and there are at most $d_i$
choices where to send each $a_i$, so there are at most $d_1d_2\cdots
d_n$ elements in $G(K/F)$.

Note: A stronger statement is true: $|G(K/F)|\leq [K:F]$. In other
words, there can never be more automorphisms that the degree of the
extension.
\end{proof}

\paragraph{Exercise 16.} Prove or disprove:
Let $f(x)$ be an irreducible cubic polynomial
in $\Q[x]$ with one real root $\alpha$. Show that the other roots form
a complex conjugate pair $\beta, \overline \beta$, so the field
$L=\Q(\beta)$ has an automorphism $\sigma$ which interchanges
$\beta,\overline \beta$.

\begin{proof}[Solution]
While the first part is true, the second part is false. Since $\beta$
is a root of the irreducible cubic polynomial $f(x)$, we
have $[\Q(\beta):\Q]=3$. So the order of $G(\Q(\beta)/\Q)$ divides
3. Now, $\sigma$ has even order, so it cannot be in $G(\Q(\beta)/\Q)$.
\end{proof}



\paragraph{Exercise 17.} Let $K$ be a Galois extension of a field $F$
such that $G(K/F)\cong C_2\times C_{12}$. How many intermediate fields
$L$ are there such that {\bf(a)} $[L:F]=4$, {\bf(b)} $[L:F]=9$,
{\bf(c)} $G(K/L)\cong C_4$.

\begin{proof}[Solution] \hfill

{\bf(a)}
By the main Galois theorem, there is exactly one intermediate field
$L$ with $[L:F]=4$ for each subgroup of $C_2\times C_{12}$ of index 4
(i.e. of order 6). Since every subgroup of $C_2\times C_{12}$ is
abelian and every abelian group of order 6 is cyclic, we only have to
check the number of cyclic subgroups of order 6. Using additive
notation with $(1,0)$ and $(0,1)$ as generators of  $C_2\times
C_{12}$,
 the three subgroups
of order 6 are generated by $(0,2)$, $(1,2)$, and $(1,4)$,
respectively. Therefore there are 3 such intermediate fields.

{\bf(b)}
Again, we are counting subgroups of index 9. Since 9 does not divide
24, there are no such intermediate fields.

{\bf(c)}
By the Galois correspondence, $G(K/L)$ is equal to the subgroup of
$G(K/F)$ that corresponds to $L$. So we need to find the number of
subgroups of $C_2\times C_{12}$ that are isomorphic to $C_4$. There
are two, generated by $(0,3)$, and $(1,3)$ respectively.
\end{proof}


\paragraph{Exercise 18.} Let $f(x)=x^4+bx^2+c\in F[x]$, and let $K$ be
the splitting field of $f$. Prove that $G(K/F)$ is contained in a
dihedral group.

\begin{proof}
By the quadratic formula, $x$ must satisfy
$$x^2=\frac{-b\pm \sqrt{b^2-4c}}{2}.$$
So the roots of $f$ are $\{\alpha,-\alpha,\beta,-\beta\}$, where
\begin{align*}
\alpha&=\sqrt{\frac{-b+\sqrt{b^2-4c}}{2}} &
\beta&= \sqrt{\frac{-b-\sqrt{b^2-4c}}{2}}.
\end{align*}
Therefore $K=F(\alpha,\beta)$. Any automorphism is completely
determined by how it acts on $\alpha$ and $\beta$, and it must send
each of $\alpha$ and $\beta$ to one of $\{\pm\alpha,\pm\beta\}$. We
have four choices for where $\alpha$ should go, and since this choice
determines where $-\alpha$ goes, we have two choices for where $\beta$
goes. So there are at most eight automorphisms of $K/F$:
\begin{align*}
\text{id}&:\text{the identity}  &
r:&\alpha\mapsto\beta\mapsto-\alpha &
r^2:&\alpha\mapsto-\alpha, \beta\mapsto-\beta&
r^3:&\alpha\mapsto-\beta\mapsto-\alpha \\
s&:\alpha\mapsto-\alpha,\beta\text{ fixed}&
sr&:\alpha\mapsto \beta\mapsto\alpha&
sr^2&:\alpha\text{ fixed}, \beta\mapsto -\beta&
sr^3&:\alpha\mapsto-\beta\mapsto\alpha
\end{align*}
This group is isomorphic to $D_4$. Of course, there may be more
relations on $\alpha$ and $\beta$ (for example perhaps $\alpha\in F$),
in which case not all of these maps correspond to elements of
$G(K/F)$. But $G(K/F)$ is a subset, and hence a subgroup of this
group, which is isomorphic to $D_4$, as desired.

It may be easier to visualize the group above as the group of
symmetries of the following diagram:
$$\xymatrix{
&\alpha \ar@{-}[dr] \ar@{-}[dl] & \\
-\beta \ar@{-}[dr] &&\beta\ar@{-}[dl]\\
&-\alpha&
}$$
Here the map $r$ is rotation clockwise by 90 degrees, and the map $s$
is a flip across the horizontal line of symmetry.




\end{proof}


\end{document}

