\documentclass[oneside,1pt]{article}

\usepackage{amsmath}

\setlength{\oddsidemargin}{0in}
\setlength{\textwidth}{6.5in}
\setlength{\topmargin}{-.5 in}
\setlength{\textheight}{8.75in}
\begin{document}

\begin{center}
{\bf Worksheet 5 --- Math 126 --- Summer 2010}
\end{center}

\noindent Consider the following two vector functions:
\begin{align*}
\mathbf r(t)&=\langle1,t+1,t\rangle \\
\mathbf q(t)&=\langle t^2,1,t\rangle
\end{align*}
for $0\leq t\leq 1$.

\begin{enumerate}
\item Suppose that these functions are the position functions for two particles. Find an equation for the distance between the two particles at a given time. 

\vfill

\item Use your Math 124 knowledge to find the times when the particles are closest together and farthest apart. Find the distances at these two times. [Hint: Remember that it is easier, but just as effective, to maximize and minimize the \emph{square} of the distance function. Also, remember to check the endpoints along with the critical points]

\vfill
\item Suppose we want to study the curves parametrized by these two functions. What function would you minimize if you want to find the place where these two curves are closest to each other? Explain why it is a function of two variables. What is the set $D$ that you are optimizing on? Ask around to make sure you have the correct function before moving on.
\vfill
\newpage

\item Find the absolute minimum and maximum values for the function you found in 3 over the set $D$. [Remember to optimize the square of the function to make things easier. Also remember to check the function along the boundaries of $D$.]

\vfill

\item Compare your answers to 4 with your answers to 2. If you did things correctly, your minimum value for 4 should be smaller than the minimum value for 1, and the maximum value for 4 should be bigger than the maximum value for 1. Why?


\hfill 

\hfill
\end{enumerate}

\end{document}

