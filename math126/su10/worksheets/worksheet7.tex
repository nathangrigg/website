% Latex document
% Math 126   Winter 2008
% First worksheet: review, TS intro 


\documentclass[12pt]{article}
\usepackage{amsmath}
\usepackage{amssymb,amsthm}
\textwidth=6.5in
\textheight=9in
\topmargin=-0.5in
\oddsidemargin= 0.0in
\evensidemargin= 0.0in



\setlength{\parskip}{1ex plus0.5ex minus0.2ex}

\pagestyle{empty}

\begin{document}

\begin{center}
\bf{Worksheet 7 --- Math 126 --- Summer 2010}
\end{center}


\vspace{0.2in}
\noindent
[1] {\it Center of Mass}.  
Find the center of mass of a lamina in the shape of an isoceles right triangle with equal sides of length $a$ if the density at any point is proportional to the square of the distance from the vertex opposite the hypotenuse.

\vspace{0.2in}
\noindent
[2] {\it Review: Tangent line approximation and tangent lines.}
Find the tangent line approximation (also called the linear approximation)
of the function $f(x) = \sqrt{1-x}$ at $a=0$, 
and use it to approximate the numbers $\sqrt{0.9}$ and $\sqrt{0.99}$.
Graph the function $f$ and the tangent line at the point $a=0$. 
Based on the picture,
determine if your estimates are above or below the actual values.

\vspace{0.2in}
\noindent
[3] \emph{Tangent Line Error Bound.}
Consider the function $\displaystyle f(x) = \int_0^x \sqrt{4-\sin(\pi t)} \, dt$.
(You should \emph{not} try to find a formula for $f$.)

(a) What is $f(0)$?   

(b) What is $f'(x)$?

(c)  Find the tangent line approximation (1st Taylor Polynomial)
     for $f(x)$ based at zero.

(d)  Use the Tangent Line Error Bound to find a bound on the error
    if you use the tangent line approximation for $f$ on the interval
    $[-1/6,1/6]$. 

(e)  Suppose you want the error to be no more than 0.01.  
     Use the Tangent Line Error Bound to find an interval around zero
     on which you can be sure the error is at most 0.01.
     Note:  The answer is not just the interval but also the argument
     showing that the bound on that interval is at most 0.01.
     The answer is not unique.

\end{document}




