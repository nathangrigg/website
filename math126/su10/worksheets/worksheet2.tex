\documentclass[12pt]{article}
%\documentstyle[12pt]{WKS6}
%\pagestyle {head}
%\pointsinmargin 
%\lhead[\bf MATH 126 Worksheet 6 Fall 2006]{Math
%126, Fall 2006}
%\chead[{}]{Worksheet 6}
%\rhead[\bf Name:\enspace\hbox to 2in{\hrulefill}]{Page \thepage}
\begin{document}
\begin{center}
{\bf Worksheet 2 --- Math 126 --- Summer 2010}
\end{center}
\noindent
The goal of this worksheet is to  get familiar with the use of polar coordinates 
and to practice the  conversion from polar coordinates to Cartesian and vice versa.   
 You should observe that some regions  are easier to understand in Cartesian coordinates whereas 
 for others the choice of polar coordinates is  much more suitable.   The skill of being 
 able to chose the right coordinate system will be invaluable when we start evaluating 
 double integrals as in chapter 15.   Later on you may wish to compare the regions you sketch today with the ones in problems 1-6, Section 15.4.
 

\vspace{0.1in}
\noindent

 The following trig identity will be useful for the worksheet: 
 $$\sin(\alpha+ \beta)= \sin(\alpha)\cos(\beta) + \cos(\alpha)\sin(\beta)$$
 $$\cos(\alpha+ \beta)= \cos(\alpha)\cos(\beta) - \sin(\alpha)\sin(\beta)$$

And here are the conversion formulas
 
$$ x = r \cos\theta, \quad y = r \sin \theta$$
$$r^2 = {x^2 + y^2}, \quad  \tan \theta = \frac{y}{x}$$

% \begin{questions}
\begin{enumerate}


\item
Describe each curve (region) below  in Cartesian coordinates.  Then sketch the curve (region).  Indicate which coordinate system you used for sketching. 
\begin{enumerate} 

%\item[(a)]  $r = \theta$, $0 \leq \theta \leq  4\pi$

%\item[(b)] $r = \sin \theta$, $0 \leq \theta \leq \pi$


\item[(a)] $r = \sin 2\theta$, $0 \leq \theta \leq \frac{\pi}{2}$


\item[(b)] $r^2 - 5r + 6 = (r-3)(r-2) < 0$  

\item[(c)] $r = \sec(\theta + \pi/3)$. Simplify your expression in Cartesian 
coordinates as much as possible, using trig identities.

\end{enumerate}

\item 
Describe each curve (region) below  in polar coordinates.  Then sketch the curve (region).  
Indicate which coordinate system you used for sketching. 
\begin{enumerate}

\item[(a)] $x = \sqrt{4-y^2}$
\item[(b)] $ x \geq 0$, $y \geq 0$, $x + y \leq 1$. Simplify your polar coordinate expression for $x+y$ to something 
involving only one trig function.  Hint:  compare to 1(c).

\end{enumerate}


\item Suppose the position of a particle as a function of time is given in 
polar coordinates as ($r(t)$, $\theta(t)$).  Find the speed of the particle, 
that is, the magnitude of the velocity vector, in terms of $r$, $\theta$, 
and their derivatives.   Hint: One way to do this is to
convert the position to Cartesian coordinates, 
use the chain rule to find the velocity, and then the speed, 
then convert the result back to polar coordinates.



\item Suppose the particle moves along one of the following curves so that 
$\theta = t$:
\begin{enumerate}
\item[(a)] $r = \theta$
\item[(b)] $r = \sin\theta$
\end{enumerate}
 Sketch the curves and find the speed of the particle in each case.  Before you do 
the calculuations, guess whether you think the particle will move with 
constant speed, and if not, where on the curve it will move faster and 
where slower.  Explain briefly the reasons for your guess. After you do 
the calculations, note whether your guess was correct.  



\item (\textit{Equations of lines in polar coordinates.}) If you have time left during the section, or later at home, 
consider the following generalization of some of the ideas introduced above.


Any line on the plane can be described by the equation  $$ax + by = d$$ 
Rewrite the equation in polar coordinates. Using the insights you
gained in 1(c) and 2(b) (and trig identity!), 
simplify   your answer  to have the form  $$r = C\sec (\theta_0 - \theta)$$  
What is the geometric meaning of the angle $\theta_0$ in your  final formula?


%\end{questions} 
\end{enumerate}
\end{document}
\end

