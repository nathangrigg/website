\magnification=\magstep1


\nopagenumbers
\centerline{\bf Worksheet 4 --- Math 126 --- Summer 2010}
\bigskip

\item{1.} 
Suppose a train follows a circular path of radius $R$ at constant speed.
Compute the acceleration, and show that it is perpendicular
to the direction of travel of the train and points into the circle.
\bigskip
\item{2.} Engineers in France observed that high--speed trains will
not stay on the tracks if the tracks are constructed of straight lines and
arcs of circles. Use the answer to problem 1 to explain why the trains
will tend to leave the tracks at the juncture of a straight track and
a circular track.

\bigskip
\item{}Remark:  If the acceleration of a vehicle changes abruptly, the passengers will feel a sudden
push in the opposite direction. For instance, if the driver steps on the accelerator, you will be
pushed backwards in your seat. Perhaps you have also felt a sideways jolt when riding in a train
or car.

\bigskip

\item{3.}
Suppose we want the train to make a smooth transition from
a straight track along the line $y = -x$
for $x \le -1$ to a straight track along the line $y = x$ for $x \ge 1$.
That is, we seek a new curve $(x, y(x))$ between (-1,1) and (1,1) that
will make the transition with no sudden changes in acceleration.
What conditions are required on the function $y(x)$ so that it connects
smoothly to the straight segments?
(Your answer should be given in terms of the values of $y$ and some of its
derivatives at $x = \pm 1$).
\bigskip
\item{4.}
In this problem you will construct a polynomial  $y(x)$ that has
the properties you listed in problem 3.

\itemitem{(a)} It's useful to look for an even polynomial, that is, one with only
even powers of $x$.  Why?

\itemitem{(b)} Find an even polynomial of degree 4 that has the properties you listed
in step 3.  (Degree 4 is suggested because you should have three conditions
at $x=1$, and an even polynomial of degree 4 has three coefficients.)
\bigskip
\item{5.} 
Find the curvature function $\kappa$ for your curve, and check that
it will vanish at both ends of the {\it curved} segment
(and thus be continuous for the entire curve,
including the straight segments).

\bigskip
\item{}Remark: Modern train tracks use circular arcs to make a
turn, but
connect them to straight segments using transition tracks (called spirals).
On the transition segments, the curvature increases from $0$
at the end attached to the straight segment, to the curvature of the circular 
arc attached at the other end.  

\end
